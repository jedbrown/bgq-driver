% begin_generated_IBM_copyright_prolog                             %
%                                                                  %
% This is an automatically generated copyright prolog.             %
% After initializing,  DO NOT MODIFY OR MOVE                       %
% ================================================================ %
%                                                                  %
% (C) Copyright IBM Corp.  2011, 2011                              %
% Eclipse Public License (EPL)                                     %
%                                                                  %
% ================================================================ %
%                                                                  %
% end_generated_IBM_copyright_prolog                               %
As supercomputers scale to a million processor cores and beyond, the underlying resource management
architecture needs to provide a flexible mechanism to manage the wide variety of workloads executing on the
machine. In this paper we describe the novel approach of the Blue Gene/Q (BG/Q) supercomputer in addressing
these workload requirements by providing resource management services that support both the high performance
computing (HPC) and high-throughput computing (HTC) paradigms. We explore how the resource management
implementations of the prior generation Blue Gene (BG/L and BG/P) systems evolved and led us down the path to
developing services on BG/Q that focus on scalability, flexibility and efficiency. Also provided is an
overview of the main components comprising the BG/Q resource management architecture and how they interact
with one another. Introduced in this paper are BG/Q concepts for partitioning I/O and compute resources to
provide I/O resiliency while at the same time providing for faster block (partition) boot times. New features, such as
the ability to run a mix of HTC and HPC workloads on the same block are explained, and the advantages of this
type of environment are examined. Similar to how Many-task computing (MTC) \cite{raicu:08} aims to combine elements of HTC
and HPC, the focus of BG/Q has been to unify the two models in a flexible manner where hybrid workloads
having both HTC and HPC characteristics are managed simultaneously.
